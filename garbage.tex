Why spin flip?

Orbital (near)-degenerate results in wave functions which are not dominated by a single configuration, but rather include several leading determinant. There fore Hartree-fock such multiconfigurational wave function is qualitatively incorrect. The traditional approach to these problem is a two step procedure . First, a zero-order wave function which consist of small of near degenerate configurations is calculated in order to recover non dynamical correlation One can use MCSCF and dynamical correlation are then recovered by configuration interaction , perturbation theory or MRCC, Recently several single- reference  method capable of describing some multi reference situation have been developed , the valence optimised orbital coupled cluster doubles method quadratic  CCD, doubley ionized  equation of motion CCSD amd spinflip approach,  With in spin flip approach  a high spin triplet state with two unpaired electron is choosen as a reference.  Target M\(_{s}\) = 0 (both closed and open-shell singlet state  and M\(_{s}\)= 0are then described as a spin-flipping e.g. \(\alpha \rightarrow \beta\), excitation from reference . In case where large nondynamical correlation drives from a single HOMO-LUMO pair, spin flip approach provides a more balanced description that the corresponding traditional single referenced method which overemphasize the importance of the closed-shell  Hartree-Fock configuration.SF is multistate approach and is therefore capable of describing several state in one calculation.With the available inexact functional however the kohn sham determinantnot only represent the total density butl also the leading electronic configuration in many electron wave function. it is there fore not surprising  that such a description can fail for systems with long multi reference character. Despite the above limitation, DFT has become very popular for its ability to describe dynamic correlation. within a computationally inexpensive one electron computational scheme. In many chemical application, its accuracy is comparable to MP2 and some time CCSD methods.This success has motivated the development od several approaches for including nondynamicla correlation within Kohn-Sham DFT . Many of these techniques adhere to a single Kohn-Sham determinant representation of the electron density.One can employ a wave function like  approach and use several kohn sham determinants for the parameterization. These determinant can be choosen by using multi reference techniques , These hybrid method combine DFT(to simulate dyanmical correlation) with multi reference models( to treal nondyanmical correlation)\\

The merge of spinflip approach with DFT proceeds through TDDFT, a formally exact single-excitation theory based on Kohn sham orbitals, TDDFT  describes excited stated as a linear response of the ground state density. TDDFT has been shown to yeild more acurate results than CIS inluding valence excited state with considerable doubley excited charecter for which CIS fail dramatically , However the TDDFT description of Rydberg stater (or diffuse valence state) is rather poor dure to the incorrect asymptotic behaviour of the limitations of ground state DFT. there fore the combination of the SF approach with TDDFT can be a simple tool to recover both non dynamical and dynamical correlation .  The attractive feature of this approach is that it is based on formally exact response theory, and therefore, SF-TDDFT will yeild exact results (also identical to non SF DFT ) if the exact functional of the density is used.

Diradicals As defined by Salem, Diradical are the molecules with two electron occupying two near degenerate molecular orbitals . while diradical are essential in interpretinf reaction mechanism. their theoretical treatment is difficult due to multiconfigurational  character of their singlet wave functions.  their high spin triplet however are single configurational  just like the triplet \(\pi\pi\ast\)

In Kohn sham DFT the electron density \(\rho(x)\) expanded over a set of M one electron (real-valued) orthonormal basis function\\ \({\phi _{p}(x)}_{p=1}^M \\
\rho(x) = \sum_{pq}^{M}\bold{P}_{pq}\phi_{p}(x)\phi_{q}(x)
\)\\
where x denotes spatial and spin coordinates of an electron \( x = (r, \sigma) \) and the density matrix \bold{P} is idempotency and normalization conditions

In TDDFT, the time-evolution of the reference state density matrix follows the time dependent Kohn-Sham equation:\\ \\
\(
[\bold{F} + \lambda \bold{V}(t), \bold{P}] = i\frac{\partial \bold{P}}{\partial t}
\)\\ \\
Where \(\lambda\bold{V}(t)\) is an infinitesimal oscillatory perturbation, and \bold{F} is the Kohn-Sham Hamiltonian matrix, which consists of the fictitious one-electron kinetic energy, nuclei-electron attraction,  electron-elctron Coulomb repulsion, and the exchange -correlation potential \(\hat{v}_{xc}\)
\\



More precise methods are required to include the effect of electron correlation on various physical observable. A strait
Traditional approach to these problems is a two step 


 

The equation of motion EOM, formalism  is an important extension of standart coupled cluster theory particularly useful in the study of excited, ionized, and electron-attached  states.  From the formal point of view the EOM-CC approach can be seen as a diagonalization of the similarity transformed  Hamiltonian, H,  in the selected configurational subspace . In this sense, the EOM formalism is very similar to the well known configuration interaction scheme in which standard hamitonian H is replace wiht \(\bar{H}\). However from the practical viewpoint the  \(\bat{H}\) is much more compicated and the construction of respective \(\bar{H}\) matrix is significantly more demanding than analogous CI matrix based on the H operator.  In practice  in both formalisms,  Davidson diagonalization  scheme is used which avoids building of the full matrix corresponding  to the considered operator within the selected configurational subspace. For all the cases, in order to achieve tractable equations, we will need to take the advantage of various factorization procedures which  formally complicate the formulas but significantly reduce the computational effort. In this paper we focus on the EE-EOM-CC  scheme based on the restricted  Hartree Fock reference and  amplitude equation in the formulation suitable for the use in generalized  Davidsom diagonalization procedure.  The conventional EE-EOM-CC scheme in the RHF formulation and the standard amplitude equations, i.e.,  setup for the spin z-component equale to zero \((S_{z} = 0)\), yeild the singlet states. The triplets  require explicit treatment of all spin cases or the additional set of the amplitude equations.  The latter case for the EE-EOM variant has been discussed in ref where the additional equation for the triplet state amplitude corresponding to the \((S_{z} = 0)\) have been implemented and applied in the RHF based triplet state calculations. We should mention that the efficient treatment of the excited state is possible also within the spin flip formalism of Krylov et al. In the current work, we develop a new approach ot the EE calculations for the high we have developed a new approach to the EE calculations for the high spin state. The approach is analogous to that established for the double electron attached case. Wo propose an independent set of the amplitude equation solved for the \(S_{z} =1 \)  (triplets and quintets) and to the case of the \(S_{z} = 0\) case.

It should be mentioned that the new equations setup for the triplet case can be used also in the multireference (MR) formulation of CC theory within the (1,1) sector of the fock space (FS).  One of the crucial step in the intermediate Hamiltonian realization of the FS-MRCC theory  with full inclusion of singles (S), doubles(D) and triples (S) MRCCSDT relies on the solution of the EE-EOM-CCSD equation with modified (dressed)\(\bar{H}\) elements. The so-called dressing is dependent only on the ground state solution hence it is identical both for singlet and triplet state. A possibility to the use the high -spin EOM scheme also in the advanced version of the MRCC fock space theory significantly widens the possible range of its application. 

As far as the case of quintets is concerned , we want to emphasize the face that the relevant equations are much simpler and easier to solve. Out of 26 diagrammatic terms contributing  to the \(R_{1}\) and \(R_{2}\) equations in the standard \((S_{z} = 0)\)  approach in the case of quintets \((S_{z} = 2)\) only \(R_{2}\) equation survives with 5 diagramatic terms present . In addition all terms  engaging three-body elements of the similarity transformed Hamiltonian disappear. The implemented method has been applied in the pilot calculations to study triplet and quintet excited state for the \(C_{2}\) molecule and the quintet states of the C and Si atoms.
\todo{This research focuses on investigation of diradical charecter of compound that are based on triqunacene as their central structure. Through this moiety, on growing cyclic ring around triqunacene various molecule can be obtained. This report focuses on theoretical studies of various molecule that could be developed from this triqunacene moiety. this compund was first reported in https://pubs.acs.org/doi/abs/10.1021/ja01069a046 in this journal  
why you did the SFTDDFT, Also about spin flip dft.

why you used the perticular basis set
why are you calculating ccsd
what is equation of motion formalism
why are you calculating eom-ccsd
what is your conclusion.}